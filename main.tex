%% TeX mode: XeLaTeX + bibTeX + XeLaTeX + XeLaTeX

%% Thesis Template of Guangzhou University
%% for using GZHUthesis package with LaTeX2e Created by Viming Wei
%% <wei_weiming@163.com> 2023-5.

\PassOptionsToPackage{quiet}{fontspec} % 消除字体警告
\expandafter\def\csname CTEX@spaceChar\endcsname{\hspace{1em}}
\documentclass[notypeinfo]{style/GZHUthesis}

% 可选参数:
% openany    新一章从奇数页或偶数页开始都可以, 去掉则必须从奇数页开始, 偶数页加空白页
% notypeinfo 取消扉页的LaTeX版本信息

%盲审模式,正常模式请注释掉
%\blindtrue
\blindfalse

%默认博士论文, 如需要硕士论文, 请注释掉第一个, 打开第二个
\phdtrue
%\phdfalse

% 添加新文件时, 需要在此处加入文件名, 方便控制编译结果
\ifblind
\includeonly{  %% 选择盲审时要编译的章节
  chapter/abstract,
  chapter/chapter1,
  chapter/chapter2,
  chapter/conclusion,
}
\else
\includeonly{  %% 选择要编译的章节, 注释掉的章节不会编译, 平时可注释掉某些章节来提升整体编译速度
  chapter/abstract,
  chapter/chapter1,
  chapter/chapter2,
  chapter/conclusion,
  chapter/publications,
  chapter/projects,
  chapter/thanks
}
\fi

% 自定义命令
% !TEX root = ../main.tex

%%%%%%%%%%%%%%%%%%%%%%%%%%% 其他设置(勿动) %%%%%%%%%%%%%%%%%%%%%%%%%%%

\ifphd
  \degree{博士}
  \englishdegree{Ph.D}
  \englishdegreea{Doctoral}
  \englishdegreeb{Doctor}
\else
  \degree{硕士}
  \englishdegree{Master}
  \englishdegreea{Masteral}
  \englishdegreeb{Master}
\fi

%% 为了方便输入特殊符号, 添加如下新命令:
\medmuskip=2mu        %水平间距调整: 二元运算符 "+, -, <"
\thickmuskip=3mu      %水平间距调整: 关系符号调整 "="
% \abovedisplayskip=0pt
% \belowdisplayskip=0pt
% \abovedisplayshortskip=0pt
% \belowdisplayshortskip=0pt

\ctexset{section={format+={\flushleft}}}  % 小节标题靠左对齐

%% 页面大小, 不要设置左右边距, 否则将出现页眉和正文无法对齐
\geometry{top=3.5cm,bottom=3.5cm}

%%%%%%%%%%%%%%%%%%%%%%%%%%%%%%%%%%%%%%%%%%%%%%%%%%%%%%%%%%%%%%%%%%

% 常用符号与自定义命令
\def\ZZ{{\mathbb Z}}
\def\NN{{\mathbb N}}
\def\RR{{\mathbb R}}
\def\CC{{\mathbb C}}
\def\QQ{{\mathbb Q}}
\def\EE{{\mathbb E}}
\def\FF{{\mathbb F}}
\def\Fp{\mathbb{F}_{p}}
\def\Fq{\mathbb{F}_{q}}
\def\Zp{\mathbb{Z}_{p}}
\def\Zq{\mathbb{Z}_{q}}
\def\Zk{\mathbb{Z}_{2^k}}
\def\Zl{\mathbb{Z}_{2^\ell}}

%\usepackage{pifont}
\newcommand{\cmark}{\ding{51}} % 打勾
\newcommand{\xmark}{\ding{55}} % 打叉

%% for algorithm
\floatname{algorithm}{算法}
\renewcommand{\algorithmicrequire}{\textbf{输入:}}
\renewcommand{\algorithmicensure}{\textbf{输出:}}

\newcommand{\upcite}[1]{\textsuperscript{\cite{#1}}} % 上标引用

% Protocol/Functionality
\tcbuselibrary{skins,breakable}
\newtcolorbox[auto counter,number within=chapter]{protocol}[2][]{%
  enhanced,
%  title        = {Protocol \thetcbcounter: {#2}},
  title        = {Protocol {#2}},
  attach boxed title to top left={xshift=+3mm,yshift*=-3mm},
  colback      = black!5,
  colframe     = black!35,
  fonttitle    = \bfseries,
  colbacktitle = black!15!white,
  arc          = 0mm,
  coltitle     = black,
  #1
}

\newtcolorbox[auto counter,number within=chapter]{functionality}[2][]{%
  enhanced,
%  title        = {Functionality \thetcbcounter: {#2}},
  title        = {Functionality {#2}},
  attach boxed title to top left={xshift=+3mm,yshift*=-3mm},
  colback      = yellow!5,
  colframe     = yellow!35!black,
  fonttitle    = \bfseries,
  colbacktitle = yellow!15!white,
  coltitle     = black,
  #1
}


% 论文基本信息设置(包含作者, 论文标题等)
% !TEX root = ../main.tex
%%%%%%%%%%%%%%%%%%%%%%%%%%%%%%
%%封面部分
%%%%%%%%%%%%%%%%%%%%%%%%%%%%%%
  
% 中文封面内容
  \title{广州大学研究生学位论文模板}
%  \version{版本:\today}
  \version{终稿}
  \subject{理学}
  \classification{O29}%
  \code{11078}
  \confidential{}
  \UDC{}
  \secrecydate{}
  \secrecyperiod{}
  \advisorinstitute{广州大学数学与信息科学学院}
  %\degree{博士}
  \major{应用数学}
  \area{密码学及其应用}
  \coverdate{2023年5月}
  \institute{广州大学数学与信息科学学院}
  \school{广州大学}
  
\ifblind
  \serialnumber{}
  \author{}
  \advisor{} %
  \supervisor{}
  \submitdate{}
  \defenddate{}
  \chairman{}
  \membera{}
  \memberb{}
  \memberc{}
  \memberd{}
\else
  \serialnumber{学号}
  \author{魏~伟~明}
  \advisor{X~X~X~~~~教授} %
  \supervisor{~~~}
  \submitdate{2023年5月29日}
  \defenddate{2023年5月29日}
  \chairman{~~~}
  \membera{~~~}
  \memberb{~~~}
  \memberc{~~~}
  \memberd{~~~}
\fi


% 英文扉页与答辩页
  \englishtitle{The Thesis Template for Guangzhou University}
  \englishsecrecydate{}
  \englishsecrecyperiod{}
  \englishsubject{Science}
  \englishinstitute{School of Mathematics and Information Science}
  \englishschool{Guangzhou University}
  \englishdegree{Doctor of Science}
  \englishmajor{Applied Mathematics}
  
\ifblind
  \englishauthor{}%
  \englishadvisor{}%
  \englishdefencedate{}
\else
  \englishauthor{Weiming Wei}%
  \englishadvisor{Prof. XXXX XXXX}%
  \englishdefencedate{May 29, 2023}
\fi



  
\begin{document}

\begin{sloppypar} % 处理行溢出
\let\standardtilde=\relax 
% 此命令可将其后所有"~"改为不可换行的空格符, 因为ctex宏包重新定义"~"是可换行的空格符
\standardtilde
\CJKspace     
% ctex宏包默认CJK*环境模式(即忽略汉字和中文标点之后的空格), 当文中数学符号比较多时很不方便.
% 解决办法: 
%         1. 正文使用 \CJKspace 切换至CJK环境模式(不忽略空格)
%         2. 或在导引中使用 \usepackage[space]{ctex}(不忽略空格). 
%         3. \begin{CJK} \end{CJK}
% 若用英文符号, 只要不在两个汉字之间"回车"换行即可.
              
\makechinesetitle  % 中文封面
%\makeenglishtitle  % 英文封面, 广大默认不使用英文封面
\makedefendpage    % 中文答辩页
\makeenglishpage   % 英文答辩页

\frontmatter  % 前言部分
 \pagenumbering{Roman} % 页码大写罗马字体
  % !TEX root = ../main.tex
\begin{abstract}
\addcontentsline{toe}{chapter}{\bfseries Abstract(In Chinese)}

广州大学研究生院只提供了Word版本的学位论文模板, 但没有提供统一的\LaTeX{}学位论文模板, 本人根据中科院学位论文模板进行了定制, 包含大多数常用的宏包, 至少数学研究生应该没有太大的问题. 希望能对后续的师弟师妹们毕业有所帮助, 不再为论文模板发愁! 

使用前, 请仔细阅读绪论, 常见问题附后. 如对模板有任何建议或意见, 欢迎发邮件联系我: \href{mailto:wei\_weiming@163.com}{wei\_weiming@163.com}. 由于工作繁忙, 没有时间修改Bug, 欢迎在Github或Gitee上提交合并. 谢谢.

\keywords{关键词1; 关键词2; 关键词3; 关键词4}
\end{abstract}

\clearpage{\cleardoublepage}
\newpage

\begin{englishabstract}
\addcontentsline{toe}{chapter}{\bfseries Abstract(In English)}

This is abstract in English.

\englishkeywords{Keyword 1; Keyword 2; Keyword 3; Keyword 4}
\end{englishabstract}
  % 摘要
  \begin{spacing}{1.3}
  \tableofcontents             % 目录
  \tableofengcontents		      % 英文目录, 只有使用双语章节时才使用, 例如\bichapter{中文章名}{English Chapter}, 此外还有\bisection, \bisubsection, \bicaption
%  \listoftables               % 表格目录
%  \listoffigures              % 插图目录
  \end{spacing}
\mainmatter   %% 正文部分

\ifblind
\linenumbers  %% 开始添加行号, 盲审时使用
\fi

% 添加新章节时必须在此处声明
% !TEX root = ../main.tex
% \chapter{绪论}
\bichapter{绪论}{Introduction} 
\label{chapter:Introduction}

%\section{模板概述} % 不需要英文目录时使用该样式
\bisection{模板概述}{Overview} % 启动双语目录必须使用此命令,其余类似

本模板基本结构如下:

\begin{itemize}
\item main.tex: 主文件, 编译对象.
\item figures: 存放论文使用的图片文件夹, 支持多种格式, 直接插入文件名, 不需要输入路径.
\item reference: 存放参考文献文件的文件夹, 默认为bib文件, 可用bibtex编译得到参考文献.
\item style: 学位论文样式文件夹, 定义了论文的整体样式.
\item chapter: 章节文件夹, 包含
\begin{itemize}
\item 论文基本信息文件(cover.cfg);
\item 自定义命令文件(defcommands.tex);
\item 摘要(abstract.tex);
\item 正文章节(chapter1.tex, chapter2.tex, conclusion.tex, ...);
\item 在读期间发表的文章(publications.tex);
\item 在读期间参与的项目(projects.tex);
\item 致谢(thanks.tex).
\end{itemize}
\end{itemize}

\bisection{编译方式}{Compilation}
本模板支持Windows/Linux/MacOS全平台编译, 亦支持Overleaf等在线编译平台, 编译方式如下:
\begin{enumerate}
\item 若本地环境请先安装texlive最新发行版, 至少为texlive 2022以上; 
\item 使用xelatex编译一次, 然后使用bibtex编译一次, 最后再次使用xelatex编译两次即可.
\end{enumerate}
推荐使用的TeX编辑器:\href{https://mirrors.tuna.tsinghua.edu.cn/github-release/texstudio-org/texstudio/}{TexStudio}, VSCode, WinEdt 11.2.

\bisection{双语目录与标题}{Bilingual Table of Contents and Titles}
有些方向要求论文有英文目录或者图表的标题是双语的, 本模板支持双语标题. 调用命令如下:
\begin{verbatim}
\bichapter{中文章节名}{English Chapter Name}
\end{verbatim}
此外, 类似的命令还有: 
\begin{itemize}
\item \verb|\bisection{中文名}{English Name}|
\item \verb|\bisubsection{中文名}{English Name}|
\item \verb|\bisubsubsection{中文名}{English Name}|
\item \verb|\bicaption{中文名}{English Name}|
\end{itemize}

为了生成双语目录, 需要在main.tex中取消\verb|\tableofengcontents|的注释.

\bisection{算法伪代码}{Algorithmic Pseudocode}
本模板已引入算法相关宏包. 使用方法见下例 \ref{alg:test}.
\begin{verbatim}
    \begin{algorithm}%[htbp]
        \caption{这是标题}
        \label{alg:test}
        \begin{algorithmic}[1] %每行显示行号
            \Require 这是输入;
            \Ensure 这是输出.
            \For{$j\leftarrow 0,\cdots, d$}
                \State {Test!} \Comment {这里是注释}
            \EndFor
            \If {$w>1$}
                \State {This is a test!}
            \EndIf
            \While {$a>1$}
                \State {Test!}
            \EndWhile
            \State {$\textbf{return }$ 0}
        \end{algorithmic}
    \end{algorithm}
\end{verbatim}

效果如下:
\begin{algorithm}%[htbp]
    \caption{这是标题}
    \label{alg:test}
    \begin{algorithmic}[1] %每行显示行号
        \Require 这是输入;
        \Ensure 这是输出.
        \For{$j\leftarrow 0,\cdots, d$}
            \State {Test!} \Comment {这里是注释}
        \EndFor
        \If {$w>1$}
        	\State {This is a test!}
        \EndIf
        \While {$a>1$}
        	\State {Test!}
        \EndWhile
        \State {$\textbf{return }$ 0}
    \end{algorithmic}
\end{algorithm}


\bisection{插图与公式}{Figures \& Equations}
如图 \ref{fig:logo}. 
\begin{figure}[h]
	\centering
	\includegraphics[scale=0.25]{logo}
	\bicaption{广州大学Logo}{GZHU LOGO}
	\label{fig:logo}
\end{figure}

如果要生成图表的标题目录, 需要在main.tex文件中取消以下两行的注释:
\begin{verbatim}
\listoftables               % 表格目录
\listoffigures              % 插图目录
\end{verbatim}

公式部分, \verb|\eqref{}|生成的公式编号带括号, 而\verb|\ref{}|生成的不带括号. 如公式 \ref{eq:eular} 或者 \eqref{eq:eular}.
\begin{equation}\label{eq:eular}
  \mathrm{e}^{i\pi}+1=0.
\end{equation}

\bisubsection{目录深度测试}{TOC Depth Test}
默认情况下目录仅显示一级标题. 可通过修改\verb|style/GZHUthesis.cls|中的如下命令中的数值来更改目录深度:
\begin{verbatim}
\setcounter{tocdepth}{1} 
\end{verbatim}

\bisection{引用参考文献}{Bibliographic Citations}
本模板使用bib文件作为参考文献文件, 同时引入了natbib宏包, 此外增添了两个自定义命令:
\begin{itemize}
\item \verb|\citeyearn{citekey}|  ~  引用年份时可在年份后面自动添加``年''
\item \verb|\upcite{citekey}| ~ 上标引用.
\end{itemize}

这是一句测试\cite{Hazay_2010_Efficient}. 引用年份时: \citeyearn{Hazay_2010_Efficient}. 上标引用\upcite{Hazay_2010_Efficient}.

\bisection{常见问题}{Q \& A}
Q1: 如何添加更多章节文件?\\
A1: 按如下步骤操作:
\begin{enumerate}
\item 在chapter文件夹下添加新的tex文件, 例如chapter3.tex
\item 打开main.tex, 仿照第28行的说明在对应位置添加\verb|chapter/chapter3,| ~ (注意后面有个英文的逗号)
\item 打开chapter3.tex, 第1行输入内容(并非一定要添加, 只是方便某些TeX编辑器识别)
\begin{verbatim}
% !TEX root = ../main.tex
\end{verbatim}
从第2行开始输入正文.
\item 编译查看结果.
\end{enumerate}

Q2: 如何修改论文作者信息?\\
A2: 直接修改cover.cfg文件中的信息.\\

Q3: 如何加入自定义命令和宏包?\\
A3: 请在defcommands.tex中添加自定义命令和宏包.\\

Q4: 可否修改为硕士学位论文?\\
A4: 本模板默认为博士学位论文. 若要修改为硕士论文, 请在main.tex文件中将语句\verb|\phdtrue|注释掉, 同时取消\verb|\phdfalse|的注释. \\

Q5: 如何生成盲审论文?\\
A5: 首先确保main.tex文件中的第28行后面这个\verb|\includeonly|命令中正确包含了需要盲审的章节, 然后按如下步骤操作:
\begin{enumerate}
\item 在main.tex文件中注释掉\verb|\blindfalse|, 打开\verb|\blindtrue|
\item 使用xelatex编译3次生成盲审结果
\end{enumerate}

Q6: 能否生成不同版本号用来区分不同的版本?\\
A6: 打开cover.cfg, 在\verb|\version{}|中填写需要的版本号, 如此, 将在首页生成一个版本号信息. 本模板提供了自定义了一个日期时间作为版本号, 调用方式为输入\verb|\version{\today}|, 结果如同(版本: \today). 当然, 版本号可以修改为任意其他文字, 例如定稿时, 修改为“终稿”.\\

Q7: 学院要求页眉只有一条细线, 不要粗线?\\
A7: 打开style/GZHUthesis.cls,搜索“粗细页眉”,注释掉相关命令。\\

Q8: 我想用\LaTeX{}默认英文字体?\\
A8: 打开style/GZHUthesis.cls,注释掉以下命令
\begin{verbatim}
\RequirePackage[nomath]{stix} % Times风格英文字体
\end{verbatim}
% !TEX root = ../main.tex
% 下面可以开始写新的正文章节, 如果要添加其他章节文件, 请务必包含上面第1句

\bichapter{第二章标题}{The Second Chapter Name}
% !TEX root = ../main.tex
\chapter{总结与展望} \label{chapter:Conclusions}

这里总结全文, 展望未来.

\appendix  % 附录

\backmatter  %% 附件部分
\bibliographystyle{style/gbt7714-2005-numerical}

% 参考文献使用 BibTeX
\bibliography{reference/myref}

% \nocite{*} %如果需要列出所有参考文献, 即使这些参考文献没有被正文引用到

\ifblind
\else
  % !TEX root = ../main.tex
\begin{publications}{99}
\ifphd
\addcontentsline{toe}{chapter}{\bf Papers Published in the Period of Doctor Education}
\else
\addcontentsline{toe}{chapter}{\bf Papers Published in the Period of Master Education}
\fi
%\thispagestyle{empty}

\item[1.]
Wang G, Tang C, \textbf{Wei W}. Some New Constructions of Optimal Asymmetric Quantum Codes[J]. \textit{Quantum Information Processing}, 2023, 22(1): 85. (SCI收录)

\item[2.]



\end{publications}
     % 发表文章目录
  % !TEX root = ../main.tex
\begin{projects}{99}
\ifphd
\addcontentsline{toe}{chapter}{\bf Projects Participated in the Period of Doctor Education}
\else
\addcontentsline{toe}{chapter}{\bf Projects Participated in the Period of Master Education}
\fi
%\thispagestyle{empty}

\item[1.]
主持广州大学研究生“基础创新”项目《黎曼猜想的若干证明》(项目编号:00001)

\item[2.]
参与国家自然科学基金项目《论补兵与守塔》(项目编号:0000001)

\item[3.]
...

\end{projects}
 % 参与项目目录
  % !TEX root = ../main.tex
\begin{thanks}
\addcontentsline{toe}{chapter}{\bf Acknowledgement}
%\thispagestyle{empty}
白驹过隙, 时光荏苒. 行文至此, 多年以来的学生时代和寒窗苦读即将落下帷幕, 再回首, 恍然如梦. 

首先, 由衷感谢我的导师唐春明教授! 

最后, 感谢在百忙之中参与评阅本文和学位答辩的各位专家和老师们! 


\vskip 6pt
\hspace{10.5cm} \textit{魏伟明}

\vskip 6pt
\hspace{9.0cm} 2023年4月于广州大学 
\end{thanks}
  % 致谢
\fi

\end{sloppypar} % 处理行溢出
\end{document} 